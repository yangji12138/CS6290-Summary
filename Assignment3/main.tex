\documentclass[conference]{IEEEtran}
\IEEEoverridecommandlockouts
% The preceding line is only needed to identify funding in the first footnote. If that is unneeded, please comment it out.
\usepackage{cite}
\usepackage{amsmath,amssymb,amsfonts}
\usepackage{algorithmic}
\usepackage{graphicx}
\usepackage{textcomp}
\usepackage{romannum}
\usepackage{enumitem}   


\def\BibTeX{{\rm B\kern-.05em{\sc i\kern-.025em b}\kern-.08em
    T\kern-.1667em\lower.7ex\hbox{E}\kern-.125emX}}
\begin{document}

\title{CS6290: Reading Summary \Romannum{3}}

\author{\IEEEauthorblockN{Yang Ji}
\IEEEauthorblockA{Dept. of Computer Science \\
ID: 56064832}
}

\maketitle

\section{Summary of Paper \cite{SongWP00}}


\subsection{Problem Statement}
This survey paper targets on a systematic summary of security and privacy issues in Bitcoin.
%
To be more specific, the authors firstly present a overview of its workflow and protocols. 
%
From perspectives of system security and privacy, they then discuss the existing vulnerabilities and countermeasures elaborately.
%
By analyzing pros and cons of these possible solutions, they summarize the open questions and latent research directions of blockchain.

\subsection{Problem Significance}
Compared to other general survey papers, this paper focuses on the security and privacy issues in Bitcoin, which renders an detailed security analysis to us. 

\subsection{State of the Art}
Apart from double spending, there are also many other types of attacks on Bitcoin like mining pool attacks, cryptography tool attacks and so on.
%
According to the tutorials taught by TA and the descriptions in paper, I conclude these attacks as follows:
%
\begin{enumerate}[label=(\roman*)]
   \item  \textbf{Double Spending.} Although Satoshi Nakamoto claimed that double spending problems can be avoided in a high probability based on the assumption of majority honest nodes.
   However, this classical problem in cryptocurrency could still happen possibly by following a specific workflow.
   For example, two partial confirmations on the same unspent money might lead to a successful double spending. 
   
   For my point of view, it is very necessary for recipients to wait seven blocks to get a global confirmation.
   But if the computational power of adversaries takes up a large proportion, waiting for multiple confirmations could still fail.
   Except for the power of CPU, some other factors like network propagation delay, exchange service connectivity and node position in network also could bring about a double-spending.
   \item  \textbf{Mining Pool Attacks.} Due to the highly concentrated power of mining pool, it could easily lead to a selfish mining or 51$\%$ attack.
   Another scenario is called Pool Hopping attack. Adversaries might collect the submitted shares from fellow miners and avoid these invalid attempts.
   What's more, bribery attacks would involve multi-party game.
   \item \textbf{Client-side Threats.} Threat model in this part is related to key management techniques.
   In particular, Bitcoin adopts elliptic curve digital signature algorithm (ECDSA) to ensure the security of transactions.
   However, this signature scheme has some potential treats (e.g. collision attack) and current system has no migration plans for broken cryptographic scheme.
   And there are many third-party wallets targeting to address the tension between usability and security at the Bitcoin client.  
   \item \textbf{Attacks on Bitcoin and Networking Infrastructure.} 
   This section lists a number of attacks on distributed network, including DDoS attacks, Malleability attacks, Refund attacks and Time jacking attack.
\end{enumerate}
%
After rendering the above four kinds of security problems, the authors also provide several possible solutions. 
%
The key problem lies in its Proof-of-Work protocol. 
%
Therefore, many variants are proposed to address potential issues from the perspective of how to generate and verify a new block.
%
Despite these variants solve some specific problems in Bitcoin, they are usually based on some strong assumptions and could also bring about new troubles.
%
For example, although Proof-of-Elapsed-Time avoids resources waste successfully, it incurs stale and broken chip problems.

Apart from security problems, privacy issues are also concerned by the public.
%
In fact, Bitcoin provides the unlinkability in some degree.
%
Users could hide their true identities in a high possibility by generating a fresh blockchain address for each transaction.
%
However, some adversaries might utilize graph analysis of transactions, addresses or entities to deduce the relationships among different transactions.
%
This heuristic method works well especially on the publicly known addresses.
%
What's more, IP address leakage could also lead to a successful deanonymization attack.

As a countermeasure, mixing protocol could solve external unlinkability effectively.
%
However, it also incurs other problems like limited scalability and internal privacy leakage.
%
Another solution, like ZeroCash and ZeroCoin, is to realize a total privacy-preserving transaction by utilizing a trusted hardware called zk-SNARKs. 

\subsection{Contributions}
This paper mainly focuses on the security and privacy issues of Bitcoin and conduct many detailed analyses on state-of-the-art solutions.

\subsection{Remaining Questions}
The domain of this paper is only limited to the security and privacy perspective of Bitcoin.
%
Other cryptocurrency like Ethereum, which firstly introduces the smart contract, has many different threats and challenges.
%
And how to effectively balance the tensions in these cryptocurrencies still remains an open question.


\section{Summary of Paper \cite{CurtmolaGKO06}}

\subsection{Problem Statement}
Guidelines: Here you should explicitly summarize the problem targeted by the paper. A rough and brief example might be as follows.
%
The paper targets the problem of searching encrypted outsourced data on the cloud.
%
Particularly, how to enable a client to perform search and get the correct search result when both the cloud database and query keywords are encrypted?

\subsection{Problem Significance}
Guidelines: Here you should explain why the problem targeted by the paper is important and has value.
%
In particular, what is the motivation of this paper?


\subsection{State of the Art}

Guidelines: Here you should describe the state of the art as reflected in the paper at that time.
%



\subsection{Contributions}
Guidelines: Here you should summarize the contributions of the paper in your own words.
%
For example, you may evaluate the contributions from the following perspectives: novelty of problem formulation, novelty of the technical solution, depth of theoretical analysis of the technical solution, positiveness of experimental evaluation result, etc.

\subsection{Remaining Questions}
Guidelines: Here you should try to discuss any remaining questions to be solved or any possible future directions based on the result of the paper.

\bibliographystyle{IEEEtran}
\bibliography{references}


\end{document}
