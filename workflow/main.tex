\documentclass[conference]{IEEEtran}
\IEEEoverridecommandlockouts
\usepackage{cite}
\usepackage{amsmath,amssymb,amsfonts}
\usepackage{algorithmic}
\usepackage{graphicx}
\usepackage{textcomp}
\usepackage{color}
\usepackage{enumerate}
\def\BibTeX{{\rm B\kern-.05em{\sc i\kern-.025em b}\kern-.08em
    T\kern-.1667em\lower.7ex\hbox{E}\kern-.125emX}}
\begin{document}

\title{CS6290: Reading Summary}

\author{\IEEEauthorblockN{Ji Yang}
\IEEEauthorblockA{ID:\ 56064832 \\
yangji@comp.hkbu.edu.hk \\
Dept.\ of Computer Science}
}

\maketitle

\section{Summary of Paper\cite{cryptobook}}

\subsection{Background}
In the first two chapters of this book\cite{cryptobook}, the authors introduce the bitcoin system structure and its workflow preliminarily.
%
And to make these concepts more understandable, they also explain some basic concepts in details.
%
Glossing over some technical details in implementation, I will summarize these two chapters from two perspectives of rudimentary concepts and system structure.

\subsection{Preliminaries}
\subsubsection{Cryptographic Hash Functions}
A cryptographic hash function $H(\cdot)$ is a mathematical function which accepts an arbitrary-length string as its input and produces a fixed-length bit string.
%
Hash function has the following three properties:

\begin{enumerate}[(i)]
    \item \textbf{Collision-resistance.} It is infeasible to find two different inputs $m_{1}$, $m_{2}$ such that $H(m_1) = H(m_2)$.
    \item \textbf{Hiding.} Given $H(m)$, there exists no efficient algorithms that can find its original input $m$.
    \item \textbf{Puzzle-friendly.} Given a uniformly chosen-random k, it is computationally infeasible to find x such that $H(k \mid x) = y$ 
\end{enumerate}

Bitcoin developers adopted a classical hash function called SHA-256 to construct other complicated data structures. Its compression feature could save data space efficiently.

\subsubsection{Merkle Hash Tree\cite{merkle1989certified}}
A Merkle Hash Tree (MHT) is a hierarchical data structure which could be used to authenticate a set of data objects with logarithmic time complexity.
%
Each leaf node contains a hash digest of a data record and each non-leaf node is assigned a hash digest from its child nodes.
%
Figure\ref{Merkle} shows an example of an MHT with four data blocks.
%
With the hash digests of its child nodes, the node $N_1$ can construct its digest  $H_{N_1} = hash(H_{N_{1-0}} \mid H_{N_{1-1}})$
%
Therefore, just keeping the root digest can ensure the integrity of whole data blocks.
%
Any manipulation on any data records will be detected immediately unless the root is changed correspondingly.

\begin{figure}[ht]
    \centering
    \includegraphics[width= 0.95\linewidth]{pic/merkle-tree.pdf}
    \caption{Merkle Hash Tree}
    \label{Merkle}
\end{figure}

\subsubsection{Digital Signature}
Basically, digital signature consists of three probabilistic polynomial time algorithms including key-generator, signing and verification.

\begin{enumerate}
    \item \textbf{Key-generator.} This step generates a public key $K_{pk}$ which is broadcast to everyone and a private key $K_{sk}$ which is kept privately by its owner.
    \item \textbf{Signing.} Given a message $m$, the prover can use own private key to construct a signature $Sig = Signing(K_{sk},m)$.
    \item \textbf{Verifying.} Given a digital signature $Sig$, a message and the prover's public key, the verifier can verdict whether this signature is valid.
          By verifying the signature, the verifier is convinced that the message is sent by the prover without any tamper. 
\end{enumerate}

In bitcoin system, public keys are also considered as identities which can be generated as needed.

\subsection{System Structure and Workflow}
Since the bitcoin network is facto distributed and peer-to-peer communication network, there are no authorities that mediate multi-party disputes.
%
To remedy the loss of the centered authority, a distributed consensus protocol must be applied in network to achieve the coordination among miners and users.
%
Each node only accepts valid transactions into its own block and other nodes accept a new block only if all transactions in this block are valid.
%
The unique data structure of blockchain can ensure the integrity of all previous blocks. 
%
Following these protocols, bitcoin system could defense some classical attacks like denial-of-service attack and double-spend attack if the majority nodes are honest.

Another significant issue in bitcoin's decentralization is incentive engineering.
%
Since bitcoin guarantees the unlinkability between accounts and real identities, we cannot reward honest nodes or penalize malicious nodes directly.
%
Instead of that, bitcoin developers adopt hash puzzles to incite miners to find a new block only by exhaustive search.
%
Once a mimer successfully finds a nonce that matches this hash puzzle and this new block is proved to be valid, he/she will get corresponding block rewards and transaction fees.

\subsection{Contributions}
The first two chapters briefly introduce the preliminaries and the way bitcoin achieves the decentralization. 


\section{Summary of Paper\cite{bonneau2015sok}}

\subsection{Problem Statement}
In recent years, blockchain technologies have gained considerable attention across the whole world due to the huge success of bitcoin. 
%
Apart from its potential effects on industry transformation, bitcoin, as a decentralized platform, also catches great attention from the academia.
%
In this trend, this paper\cite{bonneau2015sok} targets on concluding research perspectives and challenges in bitcoin from masses of scattered literatures on the Internet.
%
Through surveying the state of the art in bitcoin and its related cryptocurrencies, this paper tries to render us detailed insights from the perspectives of system structure and workflow, security analysis, privacy issues and system compatibility in bitcoin.

\subsection{Problem Significance}
It is a great challenge to capture core techniques well in cryptocurrencies due to their rapidly appearing techniques and applications.
%
What's more, how to analyze the striking issues and propose promising research directions in a reasonable manner also remains a puzzle.


\subsection{State of the Art}
In this section, I summarize this paper from the following perspectives: system structure and workflow, security analysis, privacy issues and system compatibility.

\subsubsection{System Structure and Workflow}
In the decentralized bitcoin system, the key problems are twofold. 
%
One problem is to ensure the security of public transactions since there is no centralized infrastructure like bank which can maintain the transactions service.
%
The other problem lies in how to design an efficient consensus protocol that can incentivize all users to behave honestly.

In a technical sense, some classical cryptography tools like hash function and digital signature are employed in bitcoin network.
%
For example, SHA-256 plays a role of message digests to ensure the integrity of data structures in bitcoin.
%
And users can apply Elliptic Curve Digital Signature Algorithm (ECDSA)\cite{johnson2001elliptic} to construct digital signatures with their private and public keys.
%
The whole bitcoin can be seen as a public, distributed and immutable ledger attached with all valid transactions.
%
This ledger, which is constructed as a sequence of blocks by miners, contains the following four components:
(i) Hash value of the previous block which ensures the integrity previous block; 
(ii) Timestamp; 
(iii) Consensus-proof nonce which guarantees the validity of each block;  
(iv) Merkle hash tree which aggregates the inner-block transactions efficiently.

In order to address coordination problems, bitcoin adopts the consensus mechanism called proof-of-work.
%
In this setting, miners can append a new block on the longest chain by solving a computational hash puzzle.
%
Therefore, simple exhaustive search reduces all possible competitiveness into computational power which needs ideal hardware and expensive electricity.
%
And miners are also incentivized to behave honestly by using monetary rewards which includes block rewards and transaction fees.
%
It is nothing worthy that the difficulty of new block generation is adjusted every 2016 blocks for the purpose of avoiding frequent forks.

\subsubsection{Security Analysis}
Since the security and stability of bitcoin is just proved empirically in most cases, it remains an unknown question whether it can defense against all possible novel attacks.
%
In this paper, some notions of stability and security are proposed for each component of bitcoin. Here, I try to refine the argument further and state my own understanding in details.

First, considering bitcoin is facto, decentralized, ad hoc peer-to-peer network, some security mechanisms must be deployed in the node and network level against various attacks like delay attack, cyber-attack and denial-of-service attack (DoS).
%
Therefore, bitcoin is designed as a well-connected random network with low latency for ease of rapid message diffusion. 
%
What's more, in each node level, the security mechanism will execute a screening process to refuse to accept invalid transactions like double-spending transactions.

Stability of the consensus algorithms is a more subtle problem in bitcoin.
%
All the existing analyses are based on a strong assumption that all miners will follow prescribed incentive mechanism.
%
However the realistic situation might be more complex and hard to balance perfectly.
%
There exists variety of potential problems, such as excessive power of a majority miner and collusion among smaller miners.
%
For example, in January 2019 the ethereum\cite{wood2014ethereum} suffered from a majority attack which was once deemed impossible.
%
By renting a great quantity of external computational power, attackers stole around 16 million worth of Ethereum and ERC20 tokens in this shrinking network.

\subsubsection{Privacy Issues}
In terms of users' privacy, sensitive transactions should be protected with an access control layer. 
%
However, bitcoin only provides a limited form of unlinkability.
%
By applying transactions graph analysis, adversaries can still link transactions to some specific users.
%
Even worse, all these transactions are transparent and anyone can access them.
%
Therefore, many blockchains are designated to realize this privacy issue. 
%
Zerocoin\cite{miers2013zerocoin} is the first blockchain which can provide unlinkability of transactions which also has it special coin called zerocoin.
%
Zerocash\cite{sasson2014zerocash} inherits many advantages in bitcoin and improves its efficiency further by refining underlying cryptographic operations.


\subsubsection{System Compatibility}
This section mainly involves existing and conceivable schemes of implementing changes in bitcoin.
%
In the communication network, a soft-fork change is preferred due to its backward compatibility with existing clients.
%
But sometimes, it is also essential for bitcoin developers to adopt hard-fork measurements to fill severe security gaps.
%
For instance, an anonymous hacker utilized a loophole in the coding to steal 3.6 million Ether within just few hours in July 2016. 
%
As a urgent remedy action, the Ethereum community almost unanimously agreed to implement hard-fork. 
%
In addition, alternative consensus algorithms including system parameters and computational puzzles (e.g. PoS\cite{king2012ppcoin}, PoA\cite{gunnam2008next}, Threshold delay\cite{Dfinity}, PoB\cite{Slimcoin}) are also merging constantly.  


\subsection{Contributions}
This survey introduces the primary technologies and possible future directions and provides an extensive analysis of each component in bitcoin. 
%
In general, it collects masses of scattered materials and arrange them in an innovative and reasonable manner. 

\subsection{Remaining Questions}
This survey focuses on the public blockchains like bitcoin and zerocoin. 
%
Since blockchain is a hot filed in both academia and industry, the survey could be outdated easily.
%
Nowadays, private blockchain applications like Hyperledger\cite{androulaki2018hyperledger} which adopt a deterministic protocol to ensure stability of whole system strike the world spotlight.
%
And many financial institutions consider it as a promising tool to realize an access control mechanism.


\bibliographystyle{IEEEtran}
\bibliography{references}


\end{document}
