\documentclass[conference]{IEEEtran}
\IEEEoverridecommandlockouts
% The preceding line is only needed to identify funding in the first footnote. If that is unneeded, please comment it out.
\usepackage{cite}
\usepackage{amsmath,amssymb,amsfonts}
\usepackage{algorithmic}
\usepackage{graphicx}
\usepackage{textcomp}
\def\BibTeX{{\rm B\kern-.05em{\sc i\kern-.025em b}\kern-.08em
    T\kern-.1667em\lower.7ex\hbox{E}\kern-.125emX}}
\begin{document}

\title{CS6290: Reading Summary }

\author{\IEEEauthorblockN{Student A}
\IEEEauthorblockA{Dept. of Computer Science \\
ID:12341000}
}

\maketitle

\section{Summary of Paper \cite{SongWP00}}


\subsection{Problem Statement}
Guidelines: Here you should explicitly summarize the problem targeted by the paper. A rough and brief example might be as follows.
%
The paper targets the problem of searching encrypted outsourced data on the cloud.
%
Particularly, how to enable a client to perform search and get the correct search result when both the cloud database and query keywords are encrypted?

\subsection{Problem Significance}
Guidelines: Here you should explain why the problem targeted by the paper is important and has value.
%
In particular, what is the motivation of this paper?


\subsection{State of the Art}

Guidelines: Here you should describe the state of the art as reflected in the paper at that time.
%



\subsection{Contributions}
Guidelines: Here you should summarize the contributions of the paper in your own words.
%
For example, you may evaluate the contributions from the following perspectives: novelty of problem formulation, novelty of the technical solution, depth of theoretical analysis of the technical solution, positiveness of experimental evaluation result, etc.

\subsection{Remaining Questions}
Guidelines: Here you should try to discuss any remaining questions to be solved or any possible future directions based on the result of the paper.

\section{Summary of Paper \cite{CurtmolaGKO06}}

\subsection{Problem Statement}
Guidelines: Here you should explicitly summarize the problem targeted by the paper. A rough and brief example might be as follows.
%
The paper targets the problem of searching encrypted outsourced data on the cloud.
%
Particularly, how to enable a client to perform search and get the correct search result when both the cloud database and query keywords are encrypted?

\subsection{Problem Significance}
Guidelines: Here you should explain why the problem targeted by the paper is important and has value.
%
In particular, what is the motivation of this paper?


\subsection{State of the Art}

Guidelines: Here you should describe the state of the art as reflected in the paper at that time.
%



\subsection{Contributions}
Guidelines: Here you should summarize the contributions of the paper in your own words.
%
For example, you may evaluate the contributions from the following perspectives: novelty of problem formulation, novelty of the technical solution, depth of theoretical analysis of the technical solution, positiveness of experimental evaluation result, etc.

\subsection{Remaining Questions}
Guidelines: Here you should try to discuss any remaining questions to be solved or any possible future directions based on the result of the paper.

\bibliographystyle{IEEEtran}
\bibliography{references}


\end{document}
